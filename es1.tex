\chapter{ERRORI ED ARITMETICA FINITA}
\section{Esercizio 1}
\textit{\textbf{Descrizione:} Verificare che, per h sufficientemente piccolo, $\frac{3}{2}f(x)-2f(x-h)+\frac{1}{2}(x-2h)= hf^{'}(x)+ O(h^{3})$.}\newline

\noindent\emph{Soluzione: }\newline
considerando il Polinomio di Taylor centrato in  $x_{0}$ al secondo ordine:
\begin{equation}
	 f(x_{0} - h) = f(x_{0}) - f^{1}(x_{0})h + \frac{1}{2}f^{2}(x_{0})h^{2}+O(h^{3})
\end{equation}

si effettua una sostituzione con $f(x_{0} - h)$  nella funzione e si ottiene:

\begin{equation}
  \begin{aligned}
      & \frac{2}{3}f(x) - 2\left [ f(x)-f^{1}(x)h+\frac{1}{2}f^{2}(x)h^{2}+O(h^{3}) \right ] + \\
      & +   \frac{1}{2} \left [f(x)-f^{1}(x)2h+\frac{1}{2}f^{2}(x)4h^{2}+O(h^{3})  \right ] = \\
      & = hf^{1}(x)+O(h^{3})
  \end{aligned}
\end{equation}

da cui otteniamo:

\begin{equation}
  \begin{aligned}
      & \frac{3}{2}f(x)-2f(x)+2f^{1}(x)h - \\
      &  - f^{2}(x)h^{2}+\frac{1}{2}f(x) - \\
      &  - f^{1}(x)h+f^{2}(x)h^{2}+O(h^{3}) = \\
      &  = hf^{1}(x)+O(h^{3})
  \end{aligned}
\end{equation}

quindi per h sufficientemente piccoli si ottiene:\newline

\begin{equation}
	 hf^{'}(x)+ O(h^{3}) = hf^{'}(x)+ O(h^{3})
\end{equation}
\newline
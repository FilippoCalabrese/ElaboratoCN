\section{Esercizio 7}

\textit{\textbf{Descrizione:} Calcolare la molteplicit\'a della radice nulla della funzione $f(x) = x^{2} \cdot sin(x^{2})$. Confrontare, quindi, i metodi di Newton, Newton modificato, e di Aitken, per approssimarla per gli stessi valori di tol del precedente esercizio (ed utilizzando il medesimo criterio di arresto), partendo da $x_{0}=1$. Tabulare e commentare i risultati ottenuti.}\\~\\
\emph{Soluzione:}\\~\\
La molteplicit\'a $m$ della radice nulla della funzione $f(x) = x^{2} \cdot sin(x^{2})$  \'e $m=4$. Per calcolarla, è stato utilizzato il seguente script matlab:
\\~\\

\section*{Calcolo della Molteplicità}

\lstinputlisting{resources/molteplicita.m}\newpage

\section*{metodo di  Newton modificato}

\lstinputlisting{resources/NewtonMod.m}\newpage

\section*{metodo di aitken}

\lstinputlisting{resources/aitken.m}\newpage

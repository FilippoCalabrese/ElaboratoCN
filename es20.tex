\section{Esercizio 20}

\textit
Con riferimento al precedente esercizio, tabulare il massimo errore di approssimazione (calcolato come sopra indicato), sia utilizzando le ascisse equidistanti che quelle di Chebyshev su menzionate, relativo alla spline cubica naturale interpolante $f(x)$ su tali ascisse.
\newline
\noindent\emph{Soluzione: }\newline
Utilizzando il seguente \emph{script}, tabuliamo il massimo errore di approssimazione in relazione ad \emph{n}, sia usando \emph{Ascisse Equidistanti} sia usando le \emph{Ascisse di Chebyshev}:
\section*{Tabulazione errore di interpolazione:}
\lstinputlisting{resources/20Tabulazione.m}
\newpage
Per la tabulazione, \'e stato utilizzato il seguente \emph{script}:\\
\lstinputlisting{resources/19tabulazione.m}
Di seguito riportiamo i risultati ottenuti dall'esecuzione:\\ \\
\begin{tabular}{|c|c|c|}	%	TABELLA ERRORE MASSIMO
	\hline
	n&Asc. Equidistanti&Asc. Chebyschev\\ \hline
	 2&$ 6.01194546811499 \cdot 10^{-1}   $&$ 6.01194546811499 \cdot 10^{-1} $\\ \hline
4&$ 2.79313407519679 \cdot 10^{-1}   $&$ 3.39131026109277 \cdot 10^{-1} $\\ \hline
6&$ 1.29300088354098 \cdot 10^{-1}   $&$ 2.18135003973960 \cdot 10^{-1} $\\ \hline
8&$ 5.6073852878562 \cdot 10^{-2}    $&$ 1.36254235305040 \cdot 10^{-1} $\\ \hline
10&$ 2.1973825749582 \cdot 10^{-2}    $&$ 8.2370601192499 \cdot 10^{-2} $\\ \hline
12&$ 6.908801437726 \cdot 10^{-3} $&$ 4.8074211277442 \cdot 10^{-2} $\\ \hline
14&$ 2.482863475717 \cdot 10^{-3} $&$ 2.6851489822686 \cdot 10^{-2} $\\ \hline
16&$ 3.745402833396 \cdot 10^{-3} $&$ 1.4031887250398 \cdot 10^{-2} $\\ \hline
18&$ 3.717998718041 \cdot 10^{-3} $&$ 6.489162938626 \cdot 10^{-3} $\\ \hline
20&$ 3.182857643174 \cdot 10^{-3} $&$ 2.216040957510 \cdot 10^{-3} $\\ \hline
22&$ 2.529653088972 \cdot 10^{-3} $&$ 3.078305878636 \cdot 10^{-3} $\\ \hline
24&$ 1.925792361619 \cdot 10^{-3} $&$ 3.665162885583 \cdot 10^{-3} $\\ \hline
26&$ 1.427047863663 \cdot 10^{-3} $&$ 3.755656321067 \cdot 10^{-3} $\\ \hline
28&$ 1.039053280857 \cdot 10^{-3} $&$ 3.559859767591 \cdot 10^{-3} $\\ \hline
30&$ 8.24362333267 \cdot 10^{-4} $&$ 3.218191413753 \cdot 10^{-3} $\\ \hline
32&$ 6.55498681241 \cdot 10^{-4} $&$ 2.819367883114 \cdot 10^{-3} $\\ \hline
34&$ 5.23708228635 \cdot 10^{-4} $&$ 2.416245085713 \cdot 10^{-3} $\\ \hline
36&$ 4.21003570779 \cdot 10^{-4} $&$ 2.037950397401 \cdot 10^{-3} $\\ \hline
38&$ 3.40837796214 \cdot 10^{-4} $&$ 1.698506006900 \cdot 10^{-3} $\\ \hline
40&$ 2.77976540596 \cdot 10^{-4} $&$ 1.402824430908 \cdot 10^{-3} $\\ \hline
\end{tabular}
\\ \\Nel caso proposto in questo esercizio, notiamo che la scelta delle \emph{Ascisse Equidistanti} risulta vantaggiosa rispetto alle \emph{Ascisse di Chebyshev}.
